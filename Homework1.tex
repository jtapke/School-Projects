% Options for packages loaded elsewhere
\PassOptionsToPackage{unicode}{hyperref}
\PassOptionsToPackage{hyphens}{url}
%
\documentclass[
]{article}
\usepackage{lmodern}
\usepackage{amssymb,amsmath}
\usepackage{ifxetex,ifluatex}
\ifnum 0\ifxetex 1\fi\ifluatex 1\fi=0 % if pdftex
  \usepackage[T1]{fontenc}
  \usepackage[utf8]{inputenc}
  \usepackage{textcomp} % provide euro and other symbols
\else % if luatex or xetex
  \usepackage{unicode-math}
  \defaultfontfeatures{Scale=MatchLowercase}
  \defaultfontfeatures[\rmfamily]{Ligatures=TeX,Scale=1}
\fi
% Use upquote if available, for straight quotes in verbatim environments
\IfFileExists{upquote.sty}{\usepackage{upquote}}{}
\IfFileExists{microtype.sty}{% use microtype if available
  \usepackage[]{microtype}
  \UseMicrotypeSet[protrusion]{basicmath} % disable protrusion for tt fonts
}{}
\makeatletter
\@ifundefined{KOMAClassName}{% if non-KOMA class
  \IfFileExists{parskip.sty}{%
    \usepackage{parskip}
  }{% else
    \setlength{\parindent}{0pt}
    \setlength{\parskip}{6pt plus 2pt minus 1pt}}
}{% if KOMA class
  \KOMAoptions{parskip=half}}
\makeatother
\usepackage{xcolor}
\IfFileExists{xurl.sty}{\usepackage{xurl}}{} % add URL line breaks if available
\IfFileExists{bookmark.sty}{\usepackage{bookmark}}{\usepackage{hyperref}}
\hypersetup{
  pdftitle={Chapter 1 - Introduction to Data},
  hidelinks,
  pdfcreator={LaTeX via pandoc}}
\urlstyle{same} % disable monospaced font for URLs
\usepackage[margin=1in]{geometry}
\usepackage{graphicx,grffile}
\makeatletter
\def\maxwidth{\ifdim\Gin@nat@width>\linewidth\linewidth\else\Gin@nat@width\fi}
\def\maxheight{\ifdim\Gin@nat@height>\textheight\textheight\else\Gin@nat@height\fi}
\makeatother
% Scale images if necessary, so that they will not overflow the page
% margins by default, and it is still possible to overwrite the defaults
% using explicit options in \includegraphics[width, height, ...]{}
\setkeys{Gin}{width=\maxwidth,height=\maxheight,keepaspectratio}
% Set default figure placement to htbp
\makeatletter
\def\fps@figure{htbp}
\makeatother
\setlength{\emergencystretch}{3em} % prevent overfull lines
\providecommand{\tightlist}{%
  \setlength{\itemsep}{0pt}\setlength{\parskip}{0pt}}
\setcounter{secnumdepth}{-\maxdimen} % remove section numbering
\usepackage{geometry}
\usepackage{multicol}
\usepackage{multiro}

\title{Chapter 1 - Introduction to Data}
\author{}
\date{\vspace{-2.5em}}

\begin{document}
\maketitle

\textbf{Smoking habits of UK residents}. (1.10, p.~20) A survey was
conducted to study the smoking habits of UK residents. Below is a data
matrix displaying a portion of the data collected in this survey. Note
that ``\(\pounds\)'' stands for British Pounds Sterling, ``cig'' stands
for cigarettes, and ``N/A'' refers to a missing component of the data.

\begin{center}
\scriptsize{
\begin{tabular}{rccccccc}
\hline
    & sex    & age  & marital   & grossIncome                        & smoke & amtWeekends  & amtWeekdays \\ 
\hline
1   & Female & 42   & Single    & Under $\pounds$2,600               & Yes   & 12 cig/day   & 12 cig/day \\ 
2   & Male   & 44   & Single    & $\pounds$10,400 to $\pounds$15,600 & No    & N/A          & N/A \\ 
3   & Male   & 53   & Married   & Above $\pounds$36,400              & Yes   & 6 cig/day    & 6 cig/day \\ 
\vdots & \vdots & \vdots & \vdots & \vdots                           & \vdots & \vdots      & \vdots \\ 
1691 & Male  & 40   & Single    & $\pounds$2,600 to $\pounds$5,200   & Yes   & 8 cig/day    & 8 cig/day \\   
\hline
\end{tabular}
}
\end{center}

\begin{enumerate}
\def\labelenumi{(\alph{enumi})}
\tightlist
\item
  What does each row of the data matrix represent?
\item
  How many participants were included in the survey?
\item
  Indicate whether each variable in the study is numerical or
  categorical. If numerical, identify as continuous or discrete. If
  categorical, indicate if the variable is ordinal.
\end{enumerate}

\begin{center}\rule{0.5\linewidth}{0.5pt}\end{center}

\clearpage

\textbf{Cheaters, scope of inference}. (1.14, p.~29) Exercise 1.5
introduces a study where researchers studying the relationship between
honesty, age, and self-control conducted an experiment on 160 children
between the ages of 5 and 15\footnote{Alessandro Bucciol and Marco
  Piovesan. ``Luck or cheating? A field experiment on honesty with
  children''. In: Journal of Economic Psychology 32.1 (2011), pp.~73-78.
  Available at
  \url{https://papers.ssrn.com/sol3/papers.cfm?abstract_id=1307694}}.
The researchers asked each child to toss a fair coin in private and to
record the outcome (white or black) on a paper sheet, and said they
would only reward children who report white. Half the students were
explicitly told not to cheat and the others were not given any explicit
instructions. Differences were observed in the cheating rates in the
instruction and no instruction groups, as well as some differences
across children's characteristics within each group.

\begin{enumerate}
\def\labelenumi{(\alph{enumi})}
\tightlist
\item
  Identify the population of interest and the sample in this study.
\item
  Comment on whether or not the results of the study can be generalized
  to the population, and if the findings of the study can be used to
  establish causal relationships.
\end{enumerate}

\begin{center}\rule{0.5\linewidth}{0.5pt}\end{center}

\clearpage

\textbf{Reading the paper}. (1.28, p.~31) Below are excerpts from two
articles published in the NY Times:

\begin{enumerate}
\def\labelenumi{(\alph{enumi})}
\tightlist
\item
  An article titled Risks: Smokers Found More Prone to Dementia states
  the following:
\end{enumerate}

``Researchers analyzed data from 23,123 health plan members who
participated in a voluntary exam and health behavior survey from 1978 to
1985, when they were 50-60 years old. 23 years later, about 25\% of the
group had dementia, including 1,136 with Alzheimer's disease and 416
with vascular dementia. After adjusting for other factors, the
researchers concluded that pack-a- day smokers were 37\% more likely
than nonsmokers to develop dementia, and the risks went up with
increased smoking; 44\% for one to two packs a day; and twice the risk
for more than two packs.''

Based on this study, can we conclude that smoking causes dementia later
in life? Explain your reasoning.

\begin{enumerate}
\def\labelenumi{(\alph{enumi})}
\setcounter{enumi}{1}
\tightlist
\item
  Another article titled The School Bully Is Sleepy states the
  following:
\end{enumerate}

``The University of Michigan study, collected survey data from parents
on each child's sleep habits and asked both parents and teachers to
assess behavioral concerns. About a third of the students studied were
identified by parents or teachers as having problems with disruptive
behavior or bullying. The researchers found that children who had
behavioral issues and those who were identified as bullies were twice as
likely to have shown symptoms of sleep disorders.''

A friend of yours who read the article says, ``The study shows that
sleep disorders lead to bullying in school children.'' Is this statement
justified? If not, how best can you describe the conclusion that can be
drawn from this study?

\begin{center}\rule{0.5\linewidth}{0.5pt}\end{center}

\clearpage

\textbf{Exercise and mental health.} (1.34, p.~35) A researcher is
interested in the effects of exercise on mental health and he proposes
the following study: Use stratified random sampling to ensure rep-
resentative proportions of 18-30, 31-40 and 41-55 year olds from the
population. Next, randomly assign half the subjects from each age group
to exercise twice a week, and instruct the rest not to exercise. Conduct
a mental health exam at the beginning and at the end of the study, and
compare the results.

\begin{enumerate}
\def\labelenumi{(\alph{enumi})}
\tightlist
\item
  What type of study is this?
\item
  What are the treatment and control groups in this study?
\item
  Does this study make use of blocking? If so, what is the blocking
  variable?
\item
  Does this study make use of blinding?
\item
  Comment on whether or not the results of the study can be used to
  establish a causal rela- tionship between exercise and mental health,
  and indicate whether or not the conclusions can be generalized to the
  population at large.
\item
  Suppose you are given the task of determining if this proposed study
  should get funding. Would you have any reservations about the study
  proposal?
\end{enumerate}

\end{document}
